% !TEX TS-program = pdflatex
% !TEX encoding = UTF-8 Unicode

% This is a simple template for a LaTeX document using the "article" class.
% See "book", "report", "letter" for other types of document.

\documentclass[11pt]{article} % use larger type; default would be 10pt

\usepackage[utf8]{inputenc} % set input encoding (not needed with XeLaTeX)
\usepackage{url}
\usepackage{color}
%%% Examples of Article customizations
% These packages are optional, depending whether you want the features they provide.
% See the LaTeX Companion or other references for full information.

%%% PAGE DIMENSIONS
\usepackage{geometry} % to change the page dimensions
\geometry{a4paper} % or letterpaper (US) or a5paper or....
% \geometry{margin=2in} % for example, change the margins to 2 inches all round
% \geometry{landscape} % set up the page for landscape
%   read geometry.pdf for detailed page layout information

\usepackage{graphicx} % support the \includegraphics command and options

% \usepackage[parfill]{parskip} % Activate to begin paragraphs with an empty line rather than an indent

%%% PACKAGES
\usepackage{booktabs} % for much better looking tables
\usepackage{array} % for better arrays (eg matrices) in maths
\usepackage{paralist} % very flexible & customisable lists (eg. enumerate/itemize, etc.)
\usepackage{verbatim} % adds environment for commenting out blocks of text & for better verbatim
\usepackage{subfig} % make it possible to include more than one captioned figure/table in a single float
% These packages are all incorporated in the memoir class to one degree or another...
\usepackage{url}

%%% HEADERS & FOOTERS
\usepackage{fancyhdr} % This should be set AFTER setting up the page geometry
\pagestyle{fancy} % options: empty , plain , fancy
\renewcommand{\headrulewidth}{0pt} % customise the layout...
\lhead{}\chead{}\rhead{}
\lfoot{}\cfoot{\thepage}\rfoot{}

\long\def\omitit #1{}

%%% SECTION TITLE APPEARANCE
\usepackage{sectsty}
\allsectionsfont{\sffamily\mdseries\upshape} % (See the fntguide.pdf for font help)
% (This matches ConTeXt defaults)

%%% ToC (table of contents) APPEARANCE
\usepackage[nottoc,notlof,notlot]{tocbibind} % Put the bibliography in the ToC
\usepackage[titles,subfigure]{tocloft} % Alter the style of the Table of Contents
\renewcommand{\cftsecfont}{\rmfamily\mdseries\upshape}
\renewcommand{\cftsecpagefont}{\rmfamily\mdseries\upshape} % No bold!

%%% END Article customizations

%%% The "real" document content comes below...

\title{\textbf{CMPSC 312\\ Database Systems\\Syllabus}}
\author{Spring 2022}
\date{} % Activate to display a given date or no date (if empty),
         % otherwise the current date is printed 

\tolerance=1
\emergencystretch=\maxdimen
\hyphenpenalty=10000
\hbadness=10000

\begin{document}
\maketitle


%%%%%


\subsection*{\textbf{Course Instructor}}
Dr. Oliver BONHAM-CARTER (said and written as ``Bonham-Carter,'' not “Carter'')\\
\noindent Email: \url{obonhamcarter@allegheny.edu} \\
\noindent Web Site: \url{http://www.cs.allegheny.edu/sites/obonhamcarter/} \\
\noindent Class and lab meeting place: Alden 101\\
\noindent Final Exam Code: D\\  
\noindent Final deliverable due: 18$^{th}$ May 2022 at 9:00am\\ %https://sites.allegheny.edu/registrar/fall-2021-final-exam-schedule/
\noindent Distribution Requirements: QR, SP\\
\noindent Syllabus updated on: February 22, 2022% \today\\





\subsection*{\textbf{Instructor's Office Hours}}

\begin{itemize}
  \itemsep 0em
  \item Monday and Wednesday: 11:00am -- 12:00pm (10 minute time slots)
  \item Tuesday and Thursday: 3:00pm -- 5:00pm (10 minute time slots)
  \item By appointment
\end{itemize}


\noindent
To schedule a meeting with me during my office hours, please visit my Web site and click the ``Schedule'' link in the top right-hand corner. Now, you can view my calendar or by clicking ``schedule an appointment'' link browse my office hours and schedule an appointment by clicking the correct link to reserve an open time slot. 


\subsection*{\textbf{Technical Leaders}}
	\begin{itemize}
		\item   \url{https://www.cs.allegheny.edu/teaching/technicalleaders/}
	\end{itemize}






\subsection*{\textbf{Course Meeting Schedule}}


\textbf{Lecture, Discussion, Presentations, and Group Work}:\\
\noindent
Duration: 21 Feb 2022 - 20 May 2022\\
T/Th 10:20 AM - 11:50 AM, Alden Hall, 101\\


\noindent
\textbf{Laboratory Session}:\\
Duration: 21 Feb 2022 - 20 May 2022\\
M 2:50 PM - 4:40 PM Alden Hall, 101\\





\subsection*{\textbf{Calendar}}
The calendar link is provided below to allow you to add the course and lab meeting times into your own Google calendar. Note, the whole link fits onto one line.\\
{\footnotesize
\url{https://calendar.google.com/calendar/u/0?cid=Y19hc2sxdW00MnNmaTdvcTk4YmwxOHE0M2xrNEBncm91cC5jYWxlbmRhci5nb29nbGUuY29t} }\\
Note: When copying and pasting the above hyperlink for the address, there are no spaces in the link.


\subsection*{\textbf{Discord Channel}}
The below link will expire in 7 days from 21$^{st}$ Feb 2022\\
{\footnotesize
\url{https://discord.gg/X7XXfvbX} }




\subsection*{\textbf{The {\tt ClassDocs/} Repository}}
All materials given out in class will be accessible using the {\tt classDocs/} repository. Note: The HTTP link works in absence of SSH keys.

\textbf{Main site on GitHub}: 
	\begin{itemize}
		\item \footnotesize \url{https://github.com/Allegheny-ComputerScience-312-S2022/classDocs}
	\end{itemize}

\textbf{HTTPS}: 
	\begin{itemize}
		\item {\tt \footnotesize git clone https://github.com/Allegheny-ComputerScience-312-S2022/classDocs.git}
	\end{itemize}

\textbf{SSH}: 
	\begin{itemize}
		\item {\tt \footnotesize git clone git@github.com:Allegheny-ComputerScience-312-S2022/classDocs.git}
	\end{itemize}

%%%%%






\subsection*{\textbf{Academic Bulletin Description}}

\begin{quote}

  A study of the design and implementation issues in database management systems. Topics include data models, logical/physical database design, data access/search techniques, normalization theory, mappings from logical to physical structures, storage, and utilization. Additional topics include database reorganization, migration, database integrity, consistency, privacy and security, distributed database systems, architecture of knowledge-based systems, and intelligent query interfaces. One laboratory per week. Prerequisite: Computer Science 112.  Offered in alternate years. Distribution Requirements: QR (Quantitative Reasoning), SP (Scientific Process and Knowledge).

\end{quote}


\subsection*{Distribution Requirements}
The following definitions were taken from the \emph{Distribution Requirements: Learning Outcomes} website, \url{https://sites.allegheny.edu/registrar/academic-policies/graduation-requirements/distribution-requirement/distribution-requirements-learning-outcomes/}.

\begin{itemize}
	\item \emph{Quantitative Reasoning (QR)}. Quantitative Reasoning is the ability to understand, investigate, communicate, and contextualize numerical, symbolic, and graphical information towards the exploration of natural, physical, behavioral, or social phenomena.

	\begin{itemize}
		\item Learning Outcome: Students who successfully complete this requirement will demonstrate an understanding of how to interpret numeric data and/or their graphical or symbolic representations.
	\end{itemize}


	\item \emph{Scientific Process a Knowledge (SP)}. Courses involving Scientific Process and Knowledge aim to convey an understanding of what is known or can be known about the natural world; apply scientific reasoning towards the analysis and synthesis of scientific information; and create scientifically literate citizens who can engage productively in problem solving.
	
	\begin{itemize}
		\item Learning Outcome: Students who successfully complete this requirement will demonstrate an understanding of the nature, approaches, and domain of scientific inquiry.
	\end{itemize}
\end{itemize}


This course meets the course distribution requirements of QR (Quantitative Reasoning) and SP (Scientific Process and Knowledge) for its use of applying concepts of computer programming to the design and creation databases which are tested on public data from real-world applications. In addition, the class aims to introduce an component of ethical reasoning in the design, maintenance and application of database systems for potentially sensitive data.

\subsection*{Course Objectives}

The essence of the discipline of computer science is algorithms; this course will introduce students to the principles of data management using algorithms.  We will investigate some of the key techniques that scientists use to manage data. Areas of discussion include, but are not limited to, relational databases and query languages, object-oriented data storage, encoding data in the eXtensible Markup Language (XML), low-level data storage, transactions and concurrency control, data warehousing and mining, and the implementation and testing of database applications.  \\

\noindent The course will introduce students to the theory and practice of data management while covering both the well-established and the cutting-edge areas of the discipline.  The course also invites students to assess the correctness of their implementations and conduct both analytical and empirical evaluations of the performance of data
management techniques.  Moreover, the course will ask students to implement small- and medium-scale data management systems and to install and use a wide variety of support tools. In addition to improving their teamwork skills, students will enhance their ability to write and speak about software in a clear and concise fashion. Ethical discussions are also introduced into the course to introduce students to the concepts of responsible computing. 


\subsection*{Performance Objectives}

At the completion of this class, a student must be comfortable with fundamental data management topics and be aware of current research in the area.  When given a new data management problem, students should be able to select proper data management tools and implement a complete application that uses them to solve the stated problem.  Students also must develop a toolkit of data management concepts that they can use in the context of the solutions to real-world problems. Finally, students must develop and apply a strong knowledge of analytical and empirical techniques that they can use to characterize and predict the performance of data management systems.\\

\noindent Students should also be able to handle many of the important, yet accidental, aspects of implementing programs with modern programming languages and data management systems.  In addition to being comfortable with program editors, compilers, debuggers, testing tools, virtual machines, database management systems, and query languages, students will be working with some Python programming where code will be provided to be modified.
% students should also be able to understand the purpose of shell environment variables such as the {\tt PATH} and the {\tt CLASSPATH}.


\subsection*{Required Textbooks}

% Shari Lawrence Pfleeger and Joanne M. Atlee
%   Software Engineering: Theory and Practice (Fourth Edition)
%   ISBN-10: 0136061699
%   ISBN-13: 978-0136061694
%   Prentice Hall
%   Status: Required
%   25 copies


\begin{itemize}
\item {\em Database Systems Concepts, Fifth or Sixth Edition}. Avi Silberschatz, Henry F. Korth, and S. Sudarshan.
\end{itemize}

%\subsection*{Students who want to improve their technical writing skills may consult the following books.}

%\begin{itemize}

%\item {\em BUGS in Writing: A Guide to Debugging Your Prose}. Lyn Dupr\'e. Second Edition,  ISBN-10: 020137921X,
%ISBN-13: 978-0201379211, 704 pages, 1998.

%\item {\em Writing for Computer Science}.  Justin Zobel. Second Edition,  ISBN-10: 1852338024, ISBN-13:978-1852338022, 270 pages, 2004.

%\item Along with reading the required books, you will be asked to study many additional articles from a wide variety of conference proceedings, journals, and the popular press.
%\end{itemize}


\subsection*{Class Policies}

\subsubsection*{Grading}

The grade that a student receives in this class will be based on the following categories. All percentages are
approximate and, if the need to do so presents itself, it is possible for the assigned percentages to change during the
academic semester. 
\color{red}
\begin{center}
  \begin{tabular}{l|l}
\hline

   Class Participation & 10\% \\  
    First Examination & 10\% \\
    Second Examination & 10\% \\
    Laboratory  Assignments & 40\% \\
    Final Project & 30\% \\

% no quizes this semester.
%    Class Participation & 10\% \\  %and Instructor Meetings 
%    First Quiz & 5\% \\
%    Second Quiz & 5\% \\
%    First Examination & 10\% \\
%    Second Examination & 10\% \\
%%    Final Examination & 20\% \\
%    Laboratory  Assignments & 30\% \\
%    Final Project & 30\% \\
    
    
\hline
  \end{tabular}
\end{center}
\color{black}
\noindent
These grading categories have the definitions which are defined below.
\vspace*{-.05in}


\begin{itemize}

  \item {\em Class Participation}: All students are required to actively participate during all of the class sessions. Your participation will take forms such as answering questions about the required reading assignments, completing in-class exercises, asking constructive questions of the other members of the class, giving presentations, leading a discussion session in class and in the course's Slack channels. 

%Moreover, all students are required to meet with the course instructor during office hours for a total of thirty minutes during the Fall 2014 semester.  These meetings must be scheduled through the course instructor's reservation system and documented on a meeting record that you submit on the day of the final examination. A student will receive an interim and final grade for this category.

%  \item {\em First and Second Quizzes}: The quizzes are designed to permit the student to know whether she or he is ready for the exam. Although the exams will contain new material, the quizzes will contain some of the concepts which the student may expect to see on the exam. Poor scores on quizzes will alert the student to approach the subject material with more focus.
	
  \item {\em First and Second Examinations}: The first and second examinations will cover all of the material in their associated module(s). While the second examination is not cumulative, it will assume that a student has a basic understanding of the material that was the focus of the first examination. The date for the first and second examinations will be announced at least one week in advance of the scheduled date. Unless prior arrangements are made with the course instructor, all students will be expected to take these examinations on the scheduled date and complete the tests in the stated period of time.


%  \item {\em Final Examination}: The final examination is a three-hour cumulative test. By enrolling in this course, students agree that, unless there are extenuating circumstances, they will take the final examination at the time stated on the first page of the syllabus.

  \item {\em Laboratory Assignments}: These assignments invite students to explore the concepts, tools, and techniques associated with the management of data.  All of the laboratory assignments require the use of the provided tools to design, implement, and evaluate systems that solve data management problems.  To ensure that students are ready to develop software in both other classes at Allegheny College and after graduation, the instructor will assign individuals to teams for some of the laboratory assignments.  Unless specified otherwise, each laboratory assignment will be due at the beginning of the next laboratory session.  Some of the laboratory assignments in this course will expect students to give both a short presentation and a demonstration of the software that they created to manage a collection of data.  

    %%% Homework assignments will normally ask students to prepare short written documents reflecting on
    % facets of the software development life cycles.

  \item {\em Final Project}: This project will present you with the description of a problem and ask you to implement a full-featured solution using one or more programming languages and a wide variety of data management techniques. The final project in this class will require you to apply all of the knowledge and skills that you have accumulated during the course of the semester to solve a problem and, whenever possible, make your solution publicly available as a free and open-source tool. The project will invite you to draw upon both your problem solving skills and your knowledge of programming languages and data management systems. 


\end{itemize}

\subsubsection*{Assignment Submission}

We will be using GitHub Classroom to collect all assignments. It is expected that you are able to effectively use {\tt git} to submit your work. If you require help, please see your peers, the Technology Leaders, or your instructor.\\

All assignments will have a stated due date. \color{red} The electronic version of the class assignments are to be turned in at the beginning of the lab session on that due date. Submissions after the beginning of class are counted as being late.  Assignments will be accepted for up to one week past the assigned due date with a 15\% penalty. \color{black} All late assignments must be submitted at the beginning of the session that is scheduled one week after the due date. 



\subsubsection*{\textbf{Extensions}}
Unless special arrangements are made with the course instructor, no assignments will be accepted after the late deadline. If you are requesting extensions for a lab assignment, then you are to email me with your request and also provide a \emph{valid reason} for your extension. This request must come before the due date of the lab and not on the due date. Requests will not be granted where the reason appears to be insignificant. Extensions are 24 hours of extra time (after the original due date) and are given out at my discretion. The decision to provide you with an extension (or not) will be weighed in light of fairness to your peers who are still able to complete their labs, regardless of their own busy schedules. 

The submission of homework comprises the Honor Code pledge of the student(s) completing the work. For any assignment completed in a group, students must also turn in a one-page reflection that describes each group member's contribution to the submitted deliverables.  


\subsubsection*{\textbf{Attendance}}

Classes will be attended by in-person and online students. Each class will be recorded to produce videos for online students and to enhance learning for the class.


If you will not be able to attend your session, then please email the course instructor at least one week in advance to describe your situation.  Students who miss more than five unexcused classes, laboratory sessions, or group project meetings will have their final grade in the course reduced by one letter grade. Students who miss more than ten of the aforementioned events will automatically fail the course.


\textbf{Labs}: The laboratory sessions will be held online and therefore, it is the student's responsibility to check up on materials for lab and to ask questions when necessary to ensure comprehension of deliverables. 

\subsubsection*{\textbf{Bring your own computer to class}}

The classrooms in the Department of Computer Science no longer provide machines for student use. You are to bring your own wifi-ready device to class to be able to follow along with course material. If the class is meeting online using Zoom, then please be sure that you machine is configured correctly to use these services to connect you to the class. As it is your responsibility to maintain your machine, please perform online research to determine how to configure your machine accordingly, or to install any necessary software to enable online meetings. 

During the semester, you will be told which software to install on your machine to be prepared for class. Some of the prominent software that we may be using include;

\begin{itemize}
\item Git and GiHub (a software development software system): \url{https://github.com/}
\item Atom (an editor): \url{https://atom.io/}
\item Docker (a software container system): \url{https://www.docker.com/}
	\begin{itemize}
		\item Installing Docker: \url{https://docs.docker.com/get-docker/}
		\item Basic tutorial from Docker: \url{https://www.docker.com/101-tutorial}
		\item Play with Docker: \url{https://labs.play-with-docker.com/}
	\end{itemize}
	\item SQLite (a database system): \url{https://www.sqlite.org/index.html}
	\begin{itemize}
		\item Some machines have differing methods of installing this database software. Please complete online research to read documentation to determine how to get your machine ready to run this software. 
	\end{itemize}
\end{itemize}








\vspace{-.1in}
\subsection*{\textbf{Assignment Completion}}

All assignments will have a stated due date. To accommodate for unforeseen life events, each student will be given an option of dropping one assignment grade at the end of the semester. The dropped grade cannot include the final proposal assignment. Otherwise, unless severe extenuating circumstances have been presented to the instructor, no assignments will be accepted after the deadline.

%All assignments will have a stated due date.  Late assignments will be accepted for up to one week past the assigned due date with a 15\% penalty. All late work must be submitted at the beginning of the session that is scheduled one week after the due date. Unless special arrangements are made with the course instructor, no assignments will be accepted after the late deadline.


%%%%
\omitit{
\vspace{-.10in}
\subsection*{\textbf{Laboratory Attendance}}

In order to acquire the proper skills in technical writing, critical reading, and the presentation and evaluation of technical material, it is essential for students to have hands-on experience in a laboratory. Therefore, it is mandatory for all students to attend the laboratory sessions. If you will not be able to attend a laboratory, then please see the one of the course instructor at least one week in advance in order to explain your situation. Students who miss more than two unexcused laboratories will have their final grade in the course reduced by one letter grade.  Students who miss more than four unexcused laboratories will automatically fail the course.
}%%%%







\vspace{-.10in}
\subsection*{\textbf{Extensions}}
Unless special arrangements are made with the course instructor, no assignments will be accepted after the late deadline. If you are requesting extensions for an assignment, then you are to email me with your request and also provide a \emph{valid reason} for your extension. This request must come before the due date of the lab and not on the due date. Requests will not be granted where the reason appears to be insignificant. Extensions are 24 hours of extra time (after the original due date) and are given out at my discretion. The decision to provide you with an extension (or not) will be weighed in light of fairness to your peers who are still able to complete their labs, regardless of their own busy schedules. 

\vspace{-.10in}
\subsection*{\textbf{A Note on extenuating circumstances}}

If you should find yourself in difficult circumstances that significantly interfere with your ability to prepare for this class and to complete assignments, please inform me immediately so that we can work something out together! Do not wait until the last day of class to ask for exceptions to what is stated in this syllabus. In such a situation, you may also find it helpful to contact one of the available resources on campus: 
\begin{itemize}
	\item The Maytum Learning Commons, Library/Academic Commons,\\ \url{http://sites.allegheny.edu/learningcommons/tutoring/},\\ 814-332-2898
%You may request an individual tutor through Learning Commons by visiting,

	\item Counseling \& Personal Development Center,\\ \url{https://sites.allegheny.edu/counseling/},\\ 814-332-2105
	\item Winslow Health Center,\\ \url{https://sites.allegheny.edu/healthcenter/},\\ 814-332-4355
\end{itemize}

\subsection*{\textbf{Communication}}
Various digital channels will be used in this course for communication, including email,
Discord, and the GitHub issue tracker. It is strongly advised for the student to install the Discord app on their computer and smart-phone to be sure to receive all communications from the instructor, as well as, the other members of the class.

Additionally, the course website will be used to store the syllabus, course schedule and information about the {\tt classDocs/} repository using the GitHub.  Your grades will be communicated to you by a Gradebook GitHub repository. 




\vspace*{-.10in}
\subsection*{\textbf{Special Needs and Disability Services}}

The Americans with Disabilities Act (ADA) is a federal anti-discrimination statute that provides comprehensive civil rights protection for persons with disabilities.  Among other things, this legislation requires all students with disabilities be guaranteed a learning environment that provides for reasonable accommodation of their disabilities. Students with disabilities who believe they may need accommodations in this class are encouraged to contact Disability Services at 332-2898. Disability Services is part of the Learning Commons and is located in Pelletier Library. Please do this as soon as possible to ensure that approved accommodations are implemented in a timely fashion.

\vspace{-.10in}
\subsection*{\textbf{Honor Code}}

The Academic Honor Program that governs the entire academic program at Allegheny College is described in the Allegheny Course Catalogue.  The Honor Program applies to all work that is submitted for academic credit or to meet non-credit requirements for graduation at Allegheny College.  This includes all work assigned for this class (e.g., examinations, laboratory assignments, and the final project).  All students who have enrolled in the College will work under the Honor Program.  Each student who has matriculated at the College has acknowledged the following pledge:

\vspace*{-.1in}
\begin{quote}
\emph{I hereby recognize and pledge to fulfill my responsibilities, as defined in the Honor Code, and to maintain the integrity of both myself and the College community as a whole.}
\end{quote}
\vspace*{-.15in}

\noindent It is recognized that an important part of the learning process in any course, and particularly one in computer science, derives from thoughtful discussions with teachers and fellow students.  Such dialogue is encouraged. However, it is necessary to distinguish carefully between the student who discusses the principles underlying a problem with others and the student who produces assignments that are identical to, or merely variations on, someone else's work.  While it is acceptable for students in this class to discuss their programs, technical diagrams, proposals, paper reviews, presentations, and other items with their classmates or other individuals, deliverables that are nearly identical to the work of others will be taken as evidence of violating the \mbox{Honor Code}.

















\end{document}


% junk bin %
% junk bin %
% junk bin %
% junk bin %
% junk bin %
% junk bin %






\subsubsection*{\textbf{Class Preparation}}

% The study of the computer science discipline is very challenging.  Students in this class will be challenged to learn
% the principles and practice of software development.  During the coming semester even the most diligent student will
% experience times of frustration when they are attempting to understand a challenging concept or complete a difficult
% laboratory assignment.  In many situations some of the material that we examine will initially be confusing : do not
% despair!  Press on and persevere!
% 

\noindent In order to minimize confusion and maximize learning, students must invest time to prepare for class discussions and lectures.  During the class periods, the course instructor will often pose demanding questions that could require group discussion, the creation of a program, a vote on a thought-provoking issue, or a group presentation.  Only students who have prepared for class by reading the assigned material and reviewing the current assignments will be able to effectively participate in these discussions.  More importantly, only prepared students will be able to acquire the knowledge and skills that are needed to be successful in both this course and the field of data analytics.  In order to help students remain organized and effectively prepare for classes, the course instructor will maintain a class schedule. During the class sessions students will also be required to download, write, use, and modify programs, and data sets that are made available through the course GitHub repository.


\subsubsection*{\textbf{Email}}

Using your Allegheny College email address, I will sometimes send out class announcements about matters such as assignment clarifications or changes in the schedule. It is your responsibility to check your email at least once a day and to ensure that you can reliably send and receive emails. This class policy is based on the following statement in {\em The Compass}, the college's student handbook.

\vspace*{-.1in}
\begin{quote}
  ``The use of email is a primary method of communication on campus. \ldots  All students are provided with a campus email account and address while enrolled at Allegheny and are expected to check the account on a regular
  basis.'' 
\end{quote}
\vspace*{-.1in}

\subsubsection*{\textbf{Disability Services}}

The Americans with Disabilities Act (ADA) is a federal anti-discrimination statute that provides comprehensive civil rights protection for persons with disabilities. Among other things, this legislation requires all students with disabilities be guaranteed a learning environment that provides for reasonable accommodation of their disabilities. Students with disabilities who believe they may need accommodations in this class are encouraged to contact Disability Services at (814) 332-2898.  Disability Services is part of the Learning Commons and is located in Pelletier Library. Please do this as soon as possible to ensure that approved accommodations are implemented in a timely fashion.

\subsubsection*{\textbf{Honor Code}}

The Academic Honor Program that governs the entire academic program at Allegheny College is described in the Allegheny Course Catalog and in \emph{The Compass: Student Handbook}.  The Honor Program applies to all work that is submitted for academic credit or to meet non-credit requirements for graduation at Allegheny College.  This includes all work assigned for this class (e.g., examinations, laboratory assignments, and the final project).  All students who have enrolled in the College will work under the Honor Program. Each student who has matriculated at the College has acknowledged the following pledge:

\vspace*{-.1in}
\begin{quote}
\emph{I hereby recognize and pledge to fulfill my responsibilities, as defined in the Honor Code, and to maintain the integrity of both myself and the College community as a whole.}
\end{quote}
\vspace*{-.1in}

\noindent Additionally, we expect that you will adhere to the following Department Policy:

\begin{center} \textbf{ Department of Computer Science Honor Code Policy } \end{center}
\vspace*{-.1in}
It is recognized that an important part of the learning process in any course, and particularly in computer science, derives from thoughtful discussions with teachers, student assistants, and fellow students. Such dialogue is encouraged. However, it is necessary to distinguish carefully between the student who discusses the principles underlying a problem with others, and the student who produces assignments that are identical to, or merely variations on, someone else's work. It will therefore be understood that all assignments submitted to faculty of the Department of Computer Science are to be the original work of the student submitting the assignment, and should be signed in accordance with the provisions of the Honor Code. Appropriate action will be taken when assignments give evidence that they were derived from the work of others.

%%%%
omitit{
\section*{Attendance}
% ref: https://docs.google.com/document/d/1necRH1Y9JIELUUNTYyDbvkbpH9h-n1bhhFV14A9og5w/edit


\begin{itemize}

\item \textbf{Remote Attendance}
If you are participating entirely remotely this semester and relying on technology to attend class meetings, occasional technology problems that disrupt your participation will not harm your participation grade, but as with illnesses and family emergencies, chronic absences for this reason will require a more extensive discussion with me and may impact your grade.


\item \textbf{Video and Microphones}
%In addition, faculty may want to consider a statement on \emph{netiquette} (use of microphone, video, protocols for how to participate in video conference meetings).
Please turn off your microphone when not speaking during any meeting where you are using your computer. The microphone may allow for background sound to contribute to noise during the meeting. It is strongly encouraged that you use your video to show yourself during meeting. Enabling your video will allow the instructor to see hands to indicate questions. Showing video also helps to stimulate group discussions.
\end{itemize}

}% end of omitit{}



\subsection*{College Messages}
\begin{itemize}

\item \textbf{Statement of Community}
Allegheny students and employees are committed to creating an inclusive, respectful and safe residential learning community that will actively confront and challenge racism, sexism, heterosexism, religious bigotry, and other forms of harassment and discrimination. We encourage individual growth by promoting a free exchange of ideas in a setting that values diversity, trust and equality. So that the right of all to participate in a shared learning experience is upheld, Allegheny affirms its commitment to the principles of freedom of speech and inquiry, while at the same time fostering responsibility and accountability in the exercise of these freedoms.


%\item \textbf{Academic Integrity}
%Allegheny College operates under an Honor Code, to which all students are subject. See The Compass: Student Handbook. You should educate yourself appropriately as to how this applies to you. Plagiarism and other forms of intellectual dishonesty will not be tolerated.


%\item \textbf{Disability Services}
%Students with disabilities who believe they may need accommodations in this class are encouraged to contact the Office of Disability Services at (814) 332-2898.  Disability Services is located in Pelletier Library.  Please do this as soon as possible to ensure that such accommodations are implemented in a timely fashion.


\item \textbf{Learning Commons}
If you are not already, you should become familiar with the Learning Commons, located in Pelletier Library (\url{http://sites.allegheny.edu/learningcommons/}). Among other things, the staff at the Learning Commons can assist you with study and time management skills, writing, and critical reading. You should know that if you are having trouble in this class, or if I think you can specifically benefit from their services, I will refer you to the Learning Commons. Experienced peer writing and speech consultants in the Learning Commons help writers and speakers to determine strategies for effective communication and to make academically responsible choices at any stage in the writing or speaking process and on assignments in any discipline. Both appointments and drop-in sessions are available. To view the hours of operation, and to make an appointment, visit the Learning Commons website.

\item \textbf{Religious Accommodations}
If you need to miss class or reschedule a final examination due to a religious observance, please speak to the professor well in advance to make arrangements. See \url{http://sites.allegheny.edu/religiouslife/religious-holy-days/}.
\end{itemize}

\end{document}


% junk bin %
% junk bin %
% junk bin %
% junk bin %
% junk bin %
% junk bin %







\subsection*{\textbf{Course Instructor}}
Dr. Oliver BONHAM-CARTER (said and written as ``Bonham-Carter,'' not “Carter'')\\
\noindent Classroom: Alden Hall 101 \\
\noindent Office Location: Alden Hall 104 \\
\noindent Office Phone: +1 814-332-2880 \\
\noindent Email: \url{obonhamcarter@allegheny.edu} \\
\noindent Web Site: \url{http://www.cs.allegheny.edu/sites/obonhamcarter/} \\
\noindent Slack Channel: cs312f2020\\
\noindent Slack link: \url{cs312f2020.slack.com}\\
\noindent Exam Code: J\\
\noindent Final deliverable due: 11$^{th}$ December 2020, 9:00am\\
\noindent Distribution Requirements: QR, SP\\
\noindent Syllabus updated on: \today\\


% {\color{red}Syllabus updated on: 23 August 2020}\\
% https://sites.allegheny.edu/registrar/fall-2020-final-exam-schedule/

%{\color{red} NOTE: This syllabus contains updates to respond to the college's closing due to the COVID-19 spread. These changes are in red. As always, please mail the instructor with any questions or concerns. }


\subsection*{Instructor's Office Hours}
There will be no in-person meeting times. Instead, online meetings using Zoom (download link: \url{https://zoom.us/}. To schedule a meeting with me during my office hours, please visit my web site and click the ``Schedule'' link in the top right-hand corner. Here, you can browse my office hours slots to schedule an appointment. You will find a Zoom link with the meeting slot link. At the allotted time, I will be online and awaiting your meeting. If given office hour meeting times are not convenient for your schedule, please let me know and I would be happy to work with you to find and other time which would be suitable for your schedule.


\begin{itemize}
  \itemsep 0em
  \item Tuesdays \emph{and} Thursdays: 2:00 pm -- 4:00 pm (10 minute time slots)
  \item Wednesday \emph{and} Friday: 1:30 pm -- 2:30 pm (10 minute time slots)
  \item By appointment, if these times do not work for you.
\end{itemize}

\noindent

\subsubsection*{Online Meetings}
\label{sec:meet}

We will be using Zoom to record our in-class meetings and labs for online students to participate in the course. The Zoom link for classes and the lab can be found on your shared course calendar for the event -- the calendar link will be given by a separate communication.


% Topic: CMPSC 312
% Time: This is a recurring meeting Meet anytime

% Join Zoom Meeting
% https://allegheny.zoom.us/j/99954333985

% Meeting ID: 999 5433 3985

\subsection*{Technical Leaders}
	\begin{itemize}
		\item   \url{https://www.cs.allegheny.edu/teaching/technicalleaders/}
	\end{itemize}


